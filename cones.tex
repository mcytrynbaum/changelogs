\documentclass[11pt,reqno]{amsart}
\usepackage{amssymb,mathrsfs,color}
\usepackage{pinlabel}
\usepackage{graphicx}
\usepackage{graphics} 

\graphicspath{ {c:/users/mcytrynbaum/documents/rfigures/} {c:/users/mcytrynbaum/Desktop/package/} }
\DeclareGraphicsExtensions{.pdf,.png,.jpg}

\usepackage{amsmath} % for all math functions and operations
\usepackage{amsfonts} % use this to write different scripts (e.g. Real nums, etc)
\usepackage{mathtools} %for other math stuff not included in packages above
\usepackage{amsthm} % in case you want the THM: COR: LEMMA: setup
\usepackage[top=1in,bottom=1in,left=1in,right=1in]{geometry} %for setting the margins

%\setlength\parindent{0pt}

\newtheorem{thm}{Theorem}[section]
\newtheorem{lemma}[thm]{Lemma}
\newtheorem{prop}[thm]{Proposition}
\newtheorem{cor}[thm]{Corollary}
\theoremstyle{definition}
\newtheorem{defn}[thm]{Definition}
\newtheorem{examp}[thm]{Example}
\newtheorem{remark}[thm]{Remark}
\setcounter{equation}{0}
\numberwithin{equation}{section}

\newcommand{\prf}{\begin{proof}}
\newcommand{\eprf}{\end{proof}}
\newcommand{\lft}{\left(}
\newcommand{\rt}{\right)}
\newcommand{\be}{\beta}
\newcommand{\eps}{\epsilon}
\newcommand{\wh}{\widehat}
\newcommand{\wt}{\widetilde}
\newcommand{\al}{\alpha}
\newcommand{\bp}{\begin{pmatrix}}
\newcommand{\ep}{\end{pmatrix}}
\newcommand{\inv}{^{-1}}
\newcommand{\var}{\text{Var}}
\newcommand{\cov}{\text{Cov}}
\newcommand{\corr}{\text{Corr}}
\newcommand{\ssumi}{\sum_{i=1}^n}
\newcommand{\ssumj}{\sum_{j=1}^n}
\newcommand{\ssumk}{\sum_{k=1}^n}
\newcommand{\im}{\text{Im}}
\newcommand{\mc}{\mathcal}
\newcommand{\mr}{\mathbb{R}}
\newcommand{\ol}{\overline}
\newcommand{\ul}{\underline}
\newcommand{\prob}{\mathbb{P}}
\newcommand{\ital}{\emph}
\newcommand{\tb}{\textbf}
\newcommand{\pa}{\partial}
\newcommand{\et}{\eta}
\newcommand{\argmax}{\operatornamewithlimits{argmax}}

\newcommand{\pre}{\phi}
\newcommand{\econ}{e}
\newcommand{\coordpre}{\mathrm{CP}}
\newcommand{\prealloc}{(2^X)^A}
\newcommand{\sub}{\subseteq}
\newcommand{\strcore}{\mathrm{SC}(X,U)}
\newcommand{\core}{\mathrm{C}(X,U)}
\newcommand{\stable}{\mathrm{S}(X,U)}
\newcommand{\fecon}{E}
\newcommand{\fix}{\mathcal{E}}
\newcommand{\seq}{\succeq}
\newcommand{\peq}{\preceq}
\newcommand{\su}{\succ}
\newcommand{\pe}{\prec}



\title{Using Lattice Geometry to Find All Stable Allocations}
\author{Max Cytrynbaum and Scott Duke Kominers}

\begin{document}
\maketitle

We consider a finite set of contracts $X$, each of which is associated with at least one agent $a\in A$. 
We call a subset $Y\sub X$ an \emph{allocation}, and let $2^X$ denote the set of all allocations. 
Let $d(x)$ be the set agents associated with a contract $x\in X$, and extend this definition to allocations by writing $d(Y) \equiv \bigcup_{y\in Y} d(y)$.
Note that, in general, we may have $|d(x)| > 2$ for multilateral contracts.

For $a\in A$, we let $Y_a = \{y\in Y: \, a\in d(y)\}$ denote the set of contracts associated with that agent (note that we may have $Y_a = \emptyset$). Thus, $2^{X_a}$ denotes the set of all allocations naming an agent $a\in A$. 
We assume that each $a\in A$ has strict preferences over $Y \in 2^{X_a}$, where the utility of an allocation is given by the one-to-one function $U_a: 2^{X_a} \to \mr$.
Let $\su_a$ denote the strict preference relation induced by these utility functions over bundles $Y \in 2^{X_a}$, with $\seq_a$ denoting the weak relation. Thus, $Y \seq_a  Z \iff$ $Y \su_a Z$ or $Y = Z$.

We define choice functions in the usual way for $Y\in 2^{X_a}$ by 
\[
C_a(Y) = \argmax_{Z\subseteq Y} \, U_a(Z)
\]
and, by an abuse of notation, extend choice functions to $Y\in 2^X$ by setting $C_a(Y) \equiv C_a(Y_a)$.
Throughout the matching portion of this paper, we will assume that $\emptyset_a \in X$ for all $a \in A$, where $\emptyset_a$ denotes $a$ being unmatched. Note that $d(\emptyset_a) = \{a\}$. 
\section{Finding All Stable Matchings} 
In this section, we illustrate the technique in the classical Hatfield and Milgrom model -- literature, etc... 
\subsection{Model and Notation} 
Let $A = D\times H$, where we think of $D$ as ``doctors'' and $H$ as ``Hospitals''. We may think of the set of contracts as $X = D\times H \times E$, where $E$ is a finite set of contract terms; for instance, $E$ could be a set of wages.

An allocation $Y$ is said to be \emph{stable} if it is 
\begin{enumerate}
\item \emph{individually rational} - $C_a(Y) = Y_a$ for all $a \in A$
\item \emph{unblocked} - There is no allocation $Z \not = \emptyset$ such that $Z_b \sub C_b(Y\cup Z)$ for all $b \in d(Z)$. 
\end{enumerate}
We let $S(X,U)$ denote the set of all stable matchings.

\section{Finding All Strict Core Matchings}
Talk about Echenique's contribution and what we are going to do, strict core is equivalent to...
Note how incredibly general our construction is - multilateral discrete matching in networks, essentially no structure, fully manipulates lattice structure of strict core matchings.   
\subsection{Model and Notation}
We construct an appropriate framework in which to generalize the fixed point construction in Echenique and Yenmez. 
We start by generalizing the classical notion of a \emph{prematching}. 

\begin{defn}[Preallocation] We call a map $\pre: A \to 2^X$ a \emph{preallocation} if $\pre(a) \in 2^{X_a}$ for all $a\in A$. Let $(2^X)^A$ denote the set of all preallocations.  
\end{defn}

Intuitively, a preallocation assigns each agent to a bundle of contracts naming him or her. We can think of $\pre(a)$ as the set of contracts ``held'' by agent $a$.   

We can associate each allocation $Y\subseteq X$ with a unique preallocation $\pre_Y$ in a natural way by setting $\pre_Y(a) = Y_a$.
\begin{remark} Note, however, that not all preallocations can be derived from allocations in this way.
For example, consider the case where $\emptyset \not = \pre(a)_b \not = \pre(b)_a$. In the preallocation $\pre$, $a$ holds contracts naming $b$ that are \emph{not} in the bundle of contracts held by $b$ naming $a$.  
In particular, there does not exist an allocation $Y$ such that $\pre = \pre_Y$. 
\end{remark}

With this example in mind, we say that $\pre \in \prealloc$ is a \emph{coordinated} preallocation if there exists an allocation $Y$ such that $\pre = \pre_Y$. 
We denote the set of all cooordinated preallocations by $\coordpre\sub \prealloc$.

An allocation $Y \sub X$ is said to be in the \emph{strict core} if there does not exist a blocking set $Z \sub X$ such that 
\[
U_b(Z_b) \geq U_b(Y_b) \qquad  \forall b\in d(Z)
\]
where \emph{at least one} of the above inequalities holds strictly. We denote the strict core by $SC(X,U)$. 
\subsection{Fixed Preallocations and the Strict Core}
Our method proceeds by identifying allocations $Y \in SC(X,U)$ with fixed points of an operator on preallocations, generalizing the construction in Echenique and Yenmez. 

For each agent $a \in A$, we define 

\[
V(\pre, a) = \{Z \in 2^{X_a}: \exists Y \in 2^X \, s.t. \,  Y_a = Z, \, Y_b \su_b \pre(b) \: \forall b \in d(Y)\setminus\{a\} \}
\]

Intuitively, $V(\pre, a)$ is the \emph{possibility set} for an agent $a$ at a preallocation $\pre$. 
It contains all sets of contracts naming $a$ that are part of a larger economy $Y$ where every other agent $b \in d(Y)$ weakly prefers their contracts under $Y$ to their contracts under the pre-allocation $\pre$.  

Next, we define an operator $T: \prealloc \to \prealloc$ by setting $T \pre(a) = \max V(\pre, a)$, where the maximum is taken under the preference relation $\seq_a$ for each $a \in A$.
Note that $\emptyset_a \in V(\pre,a)$ for any $\pre \in \prealloc$, so $T$ is well-defined. Let $\fix(T)$ denote the fixed points of $T$. Define $\fecon(T) = \{Y \in 2^X: \, \pre_Y \in \fix(T)\}$, the collection of allocations $Y$ whose corresponding preallocation $\pre_Y$ is fixed by $T$.

Before our first result, we note a simple fact: if $\pre \in \coordpre$, then $\pre(a) \in V(\pre, a)$.
To see this, note that $\pre \in \coordpre$ means that there exists $Y \sub X$ with $\pre = \pre_Y$. 
Then $Y$ is an allocation satisfying the conditions in $V(\pre_Y,a)$, so that $Y_a  = \pre(a) \in V(\pre,a)$. 
With the definitions above, we have a simple result
\begin{lemma} $\fecon(T) = \strcore$
\end{lemma}
\prf
First, suppose that $Y \not \in \strcore$. Then by definition, there exists some blocking allocation $\emptyset \not = Z \sub X$ such $Z_b \seq_b Y_b$ for all $b \in d(Z)$.
Let $a \in d(Z)$ be such that the inequality above is strict, and consider $\pre = \pre_Y$. 
In particular, $Z_b \seq_b \pre_Y(b)$ for all $b \in d(Z) \setminus \{a\}$, so that $Z_a \in V(\pre_Y, a)$.
Then $T \pre_Y(a) = \max V(\pre_Y, a) \seq_a Z_a \su_a Y_a = \pre_Y(a)$, so $T\pre_Y(a) \not = \pre_Y(a)$, and $Y \not \in \fecon(T)$.  

Suppose, conversely, that $Y \not \in \fecon(T)$ so that $\pre_Y \not \in \fix(T)$.  
Then there exists an agent $a \in A$ such that $T\pre_Y(a) = Z_a \not = \pre_Y(a)$ for some allocation $Z$.  
By the definition of $V(\pre_Y,a)$, we have $Z_b \seq_b \pre_Y(b) = Y_b$ for $b \in d(Z) \setminus \{a\}$.
We know $\pre_Y$ is coordinated, so by the simple fact above $\pre_Y(a) \in V(\pre_Y,a)$.
Then $Z_a = T \pre_Y(a) \su_a \pre_Y(a) =  Y_a$. Then $Z$ is a blocking coalition for $Y$, so $Y \not \in \strcore$.  
\eprf
Thus, we have identified $\strcore$ with the set of \emph{coordinated} preallocations $\pre \in \coordpre$ such that $T \pre = \pre$. 
This result shows that an algorithm that finds all $\pre \in \fix(T)$ will also find all strict core matchings. 

However, if there are \emph{uncoordinated} preallocations that are also fixed by $T$, such an algorithm may return extraneous solutions not associated with any strict core matching.
The following lemma, which is essential for the construction of our algorithm, shows that there are no such preallocations.
Note that this result significantly generalizes the corresponding lemma in Echenique and Yenmez and also subsumes the main theorem of Kojima (xxxx). 

\begin{lemma} $\fix(T) \sub \coordpre$
\end{lemma}
\prf
We begin with an important fact that will be used repeatedly.
Suppose that $\pre \in \fix(T)$. Then for any $a \in A$, we have $\pre(a) = T\pre(a) \in V(\pre,a)$.
Thus, there exists an allocation $Y$ such that $Y_a = \pre(a)$, and $Y_b \seq_b \pre(b)$ for all $b \in d(Y) \setminus \{a\}$.
Since $Y_a = \pre(a)$, then in fact $Y_b \seq_b \pre(b)$ holds \emph{for all} agents $b \in d(Y)$. 
For any $b \in d(Y)$, $Y$ then satisfies the conditions in the definition of $V(\pre,b)$, so $Y_b \in V(\pre,b)$. 
Therefore, $\pre(b) = T\pre(b) \seq Y_b \seq \pre(b)$, so equality holds throughout. In particular, $Y_b = \pre(b)$ for all $b \in d(Y)$. 

Fix $a_1 \in A$. Since $\pre \in \fix(T)$, the argument above shows that there exists an allocation $Y$ such that $Y_{a_1} = \pre(a_1)$, and, in particular, $Y_b = \pre(b)$ for all $b \in d(Y)$. 
Therefore, $U(\pre,a_1) = \{Y \in 2^X: Y_{a_1} = \pre(a_1), \: Y_b \seq \pre(b)\: \forall b \in d(Y)\setminus \{a_1\}\}$, the collection of global allocations available to $a_1$ at $\pre$, is non-empty, so there exists an allocation 

\[
Y \in \argmax_{Z \in U(\pre,a_1)} |d(Z)|
\]

Let $A_1 = d(Y)$. If $A_1 = A$, we are done, since then by the construction above $Y_a = \pre(a)$ for all $a \in A$, so $\pre = \pre_Y$ and $\pre \in \coordpre$. 

Then assume that $A_1 \not = A$, and pick $a_2 \in A \setminus A_1$. By the fact at the beginning of the proof, there exists an allocation $Z$ such that $Z_{a_2} = \pre(a_2)$, and, in fact, $\pre(b) = Z_b$ for all $b \in d(Z)$. 
Define $A_2 = d(Z) \cap A_1^c$, which is non-empty by construction.
Let $b \in A_2$. We will show that $d(\pre(b)) \cap A_1 = \emptyset$.
That is, under the preallocation $\pre$, agent $b \in A_2$ is \emph{not} holding any contracts that name agents in $A_1$. 

Suppose not, so there exists $c \in A_1 \cap d(\pre(b))$.
Then, in particular, $c \in d(\pre(b)) = d(Z_b) \sub d(Z)$, so applying the fact proved at the beginning, $Z_c = \pre(c) = Y_c$. 
Since $c \in d(Z_b)$, there exists a contract $z \in Z_c$ naming both $c$ and $b$.
Then $b \in Z_c = Y_c$, so $b \in d(Y_c) \sub d(Y) = A_1$, so $b \in A_1 \cap A_2 = \emptyset$. 
This is a contradiction, so it must be the case that $d(\pre(b)) \cap A_1 = \emptyset$ for all $b \in A_2$. 

Define $S = \bigcup_{b \in A_2} \pre(b)$. We have just shown that $d(S) \sub A_1^c$.
We also have $d(S) = \bigcup_{b \in A_2} d(\pre(b)) = \bigcup_{b \in A_2} d(Z_b) \sub d(Z)$, so $d(S) \sub A_1^c \cap d(Z) = A_2$.
Clearly $b \in d(\pre(b))$ for all $b \in A_2$, so $A_2 \sub d(S)$. Then $A_2 = d(S)$. 

Set $W = Y \cup S$. We have now shown that $A_2 \not = \emptyset$ and $A_1 \cap A_2 = \emptyset$.
Since $d(Y) = A_1$ and $d(W) = A_2$, it follows that $W \cap Y = \emptyset$, so we have  

\begin{enumerate}
\item $W_b = Y_b = \pre(b)$ for all $b \in A_1$; in particular, $W_{a_1} = \pre(a_1)$. 
\item $W_b = S_b = \pre(b)$ for all $b \in A_2$.
\end{enumerate}

Then apparently $W \in U(\pre,a_1)$ as defined above. However, by construction $|d(W)| > |d(Y)|$, which contradicts our original choice of $Y$.
This finishes the proof.
\eprf
\subsubsection{Discussion}
Combining these lemmas, we see that searching for strict core allocations in a very general model of multilateral matching with contracts is equivalent to searching for the fixed points of $T$. 
Our algorithm depends heavily on this result, which shows, critically, that the fixed points $\fix(T)$ are only as dense in $\prealloc$ as the strict core outcomes.

Our maximal domain results will show that this is not the case for the natural extension of this method to \emph{true} core outcomes $\core$.
For true core outcomes, where the lattice algorithm fails, $T$ also fixes at a large number of extraneous, uncoordinated preallocations. 
\subsection{The Lattice of Fixed Preallocations}
In this section, we generalize constructions from Echenqiue and Yenmez showing the the fixed points of the squared operator $T^2$ form a lattice. 
First, we define a natural partial order on the set of preallocations $\prealloc$.

Say that $\pre \su \pre'$ if and only if $\pre(a) \seq_a \pre'(a)$ for all $a \in A$, where at least one of these inequalities \emph{holds strictly}. 
Thus, we write $\pre \seq \pre'$ if and only if $\pre \su \pre'$ or $\pre = \pre'$. 
This is a product order on a product space, which makes $\prealloc$ into a complete lattice (cite)(footnote about joins and meets). 
Next, we give a sequence of results concerning the operator $T$ and its fixed points. 
These results are an almost direct extension Lemmas 4.x through 4.y of Echenique and Yenmez.
For convenience, we reproduce the first result in our framework - the rest follow from work in Echenique in and Yenmez. 
\begin{lemma} $T$ is antitone 
\end{lemma}
\prf
Let $\pre \leq \pre'$ be preallocations.
Fix $a \in A$, and let $Z \in V(\pre',a)$.
Then there is an allocation $Y \sub X$ with $Y_a = Z$ such that $Y_b \seq_b \pre'(b)$ for $b \in d(Y) \setminus \{a\}$.
Then $Y_b \seq_b \pre'(b) \seq_b \pre_b$ also for all such agents, so we also have $Z \in V(\pre,a)$.
Then $V(\pre,a) \supseteq V(\pre',a)$, so that $T\pre(a) \seq T\pre'(a)$. $a$ was arbitrary, so $T\pre \seq T\pre'$ under our partial order. 
\eprf
The following lemmas follow exactly as in Echenique and Yenmez, using the antitonicity of $T$. 
\begin{cor} $T^2$ is isotone, and $\fix(T^2)$ is a non-empty complete lattice. 
\end{cor}
\begin{lemma} No two preallocations $\pre$ and $\pre'$ can be compared under the partial order on $\prealloc$.
\end{lemma}
\begin{lemma} There exist preallocations $\ol{\pre}$ and $\ul{\pre}$ such that for all $\pre \in \fix(T)$, we have $\ol{\pre} \seq \pre \seq \ul{\pre}$. Moreover, if $\pre = \ol{\pre}$ or $\pre = \ul{\pre}$, then $\fix(T) = \{\pre\}$. 
\end{lemma}

\subsection{Exploiting Lattice Geometry to Find All Strict Core Allocations}
In this section, we improve upon the algorithm of Echenique (2003) and Echenique and Yenmez (2013), showing how to fully exploit the geometry of the lattice $\fix(T)$ to more quickly find all strict core matchings. 






\subsubsection{Discussion}
As can be seen, the construction above is very general, requiring only that the fixed points we want to find are bounded by a lattice, and that we have a way to initialize new problems on sublattices whose solution sets bound the solution set of our original problem. 

\section{Finding All Nash Equilibria in Games with Strategic Complements}
\subsection{Model and Notation}









\end{document}

