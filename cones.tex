\documentclass[11pt,reqno]{amsart}
\usepackage{amssymb,mathrsfs,color}
\usepackage{pinlabel}
\usepackage{graphicx}
\usepackage{graphics} 

\graphicspath{ {c:/users/mcytrynbaum/documents/rfigures/} {c:/users/mcytrynbaum/Desktop/package/} }
\DeclareGraphicsExtensions{.pdf,.png,.jpg}

\usepackage{amsmath} % for all math functions and operations
\usepackage{amsfonts} % use this to write different scripts (e.g. Real nums, etc)
\usepackage{mathtools} %for other math stuff not included in packages above
\usepackage{amsthm} % in case you want the THM: COR: LEMMA: setup
\usepackage[top=1in,bottom=1in,left=1in,right=1in]{geometry} %for setting the margins

%\setlength\parindent{0pt}

\newtheorem{thm}{Theorem}[section]
\newtheorem{lemma}[thm]{Lemma}
\newtheorem{prop}[thm]{Proposition}
\newtheorem{cor}[thm]{Corollary}
\theoremstyle{definition}
\newtheorem{defn}[thm]{Definition}
\newtheorem{examp}[thm]{Example}
\newtheorem{remark}[thm]{Remark}
\setcounter{equation}{0}
\numberwithin{equation}{section}

\newcommand{\prf}{\begin{proof}}
\newcommand{\eprf}{\end{proof}}
\newcommand{\lft}{\left(}
\newcommand{\rt}{\right)}
\newcommand{\be}{\beta}
\newcommand{\eps}{\epsilon}
\newcommand{\wh}{\widehat}
\newcommand{\wt}{\widetilde}
\newcommand{\al}{\alpha}
\newcommand{\bp}{\begin{pmatrix}}
\newcommand{\ep}{\end{pmatrix}}
\newcommand{\inv}{^{-1}}
\newcommand{\var}{\text{Var}}
\newcommand{\cov}{\text{Cov}}
\newcommand{\corr}{\text{Corr}}
\newcommand{\ssumi}{\sum_{i=1}^n}
\newcommand{\ssumj}{\sum_{j=1}^n}
\newcommand{\ssumk}{\sum_{k=1}^n}
\newcommand{\im}{\text{Im}}
\newcommand{\mc}{\mathcal}
\newcommand{\mr}{\mathbb{R}}
\newcommand{\ol}{\overline}
\newcommand{\ul}{\underline}
\newcommand{\prob}{\mathbb{P}}
\newcommand{\ital}{\emph}
\newcommand{\tb}{\textbf}
\newcommand{\pa}{\partial}
\newcommand{\et}{\eta}

\newcommand{\pre}{\phi}
\newcommand{\econ}{e}
\newcommand{\coordpre}{CP}
\newcommand{\prealloc}{(2^X)^A}
\newcommand{\sub}{\subseteq}
\newcommand{\strcore}{SC(X,U)}
\newcommand{\core}{C(X,U)}
\newcommand{\stable}{S(X,U)}
\newcommand{\fecon}{E}
\newcommand{\fix}{\mathcal{E}}


\title{Using Lattice Geometry to Find All Stable Allocations}
\author{Max Cytrynbaum and Scott Duke Kominers}

\begin{document}
\maketitle

We consider a finite set of contracts $X$, each of which is associated with at least one agent $a\in A$. 
We call a subset $Y\sub X$ an \emph{allocation}, and let $2^X$ denote the set of all allocations. 
Let $d(x)$ be the set agents associated with a contract $x\in X$, and extend this definition to allocations by writing $d(Y) \equiv \bigcup_{y\in Y} d(y)$. Note that, in general, we may have $|d(x)| > 2$ for multilateral contracts.

For $a\in A$, we let $Y_a = \{y\in Y: \, a\in d(y)\}$ denote the set of contracts associated with that agent (note that we may have $Y_a = \emptyset$). Thus, $2^{X_a}$ denotes the set of all allocations naming an agent $a\in A$. 
We assume that each $a\in A$ has strict preferences over $Y \in 2^{X_a}$, where the utility of an allocation is given by the one-to-one function $U_a: 2^{X_a} \to \mr$. Let $>_a$ denote the strict preference relation induced by these utility functions over bundles $Y \in 2^{X_a}$, with $\geq_a$ denoting the weak relation. Thus, $Y \geq_a  Z \iff$ $Y >_a Z$ or $Y = Z$.

We define choice functions in the usual way for $Y\in 2^{X_a}$ by 
\[
C_a(Y) = \max_{Z\subseteq Y} \, U_a(Z)
\]
and, by an abuse of notation, extend choice functions to $Y\in 2^X$ by setting $C_a(Y) \equiv C_a(Y_a)$.
Throughout the matching portion of this paper, we will assume that $\emptyset_a \in X$ for all $a \in A$, where $\emptyset_a$ denotes $a$ being unmatched. Note that $d(\emptyset_a) = \{a\}$. 
\section{Finding All Stable Matchings} 
In this section, we illustrate the technique in the classical Hatfield and Milgrom model -- literature, etc... 
\subsection{Model and Notation} 
Let $A = D\times H$, where we think of $D$ as ``doctors'' and $H$ as ``Hospitals''. We may think of the set of contracts as $X = D\times H \times E$, where $E$ is a finite set of contract terms; for instance, $E$ could be a set of wages.

An allocation $Y$ is said to be \emph{stable} if it is 
\begin{enumerate}
\item \emph{individually rational} - $C_a(Y) = Y_a$ for all $a \in A$
\item \emph{unblocked} - There is no allocation $Z \not = \emptyset$ such that $Z_b \sub C_b(Y\cup Z)$ for all $b \in d(Z)$. 
\end{enumerate}
We let $S(X,U)$ denote the set of all stable matchings.

\section{Finding All Strict Core Matchings}
Talk about Echenique's contribution and what we are going to do, strict core is equivalent to...
Note how incredibly general our construction is - multilateral discrete matching in networks, essentially no structure, fully manipulates lattice structure of strict core matchings.   
\subsection{Model and Notation}
We construct an appropriate framework in which to generalize the fixed point construction in Echenique and Yenmez. 
We start by generalizing the classical notion of a \emph{prematching}. 

\begin{defn}[Preallocation] We call a map $\pre: A \to 2^X$ a \emph{preallocation} if $\pre(a) \in 2^{X_a}$ for all $a\in A$. Let $(2^X)^A$ denote the set of all preallocations.  
\end{defn}

Intuitively, a preallocation assigns each agent to a bundle of contracts naming him or her. We can think of $\pre(a)$ as the set of contracts ``held'' by agent $a$.   
For a preallocation $\pre$, we define the \emph{economy} associated with $\pre$ by $\econ(\pre) \equiv \bigcup_{a\in A}\pre(a)$ 

We can associate each allocation $Y\subseteq X$ with a unique preallocation $\pre_Y$ in a natural way by setting $\pre_Y(a) = Y_a$.
\begin{remark} Note, however, that not all preallocations can be derived from allocations in this way.
For example, consider the case where $\emptyset \not = \pre(a)_b \not = \pre(b)_a$. In the preallocation $\pre$, $a$ holds contracts naming $b$ that are \emph{not} in the bundle of contracts held by $b$ naming $a$.  
In particular, there does not exist an allocation $Y$ such that $\pre = \pre_Y$. 
\end{remark}

With this example in mind, we say that $\pre \in \prealloc$ is a \emph{coordinated} preallocation if there exists an allocation $Y$ such that $\pre = \pre_Y$. 
We denote the set of all cooordinated preallocations by $\coordpre\sub \prealloc$.
Note that $\pre \in \coordpre$ if and only if $ \pre = \pre_{\econ(\pre)}$

An allocation $Y \sub X$ is said to be in the \emph{strict core} if there does not exist a blocking set $Z \sub X$ such that 
\[
U_b(Z_b) \geq U_b(Y_b) \qquad  \forall b\in d(Z)
\]
where \emph{at least one} of the above inequalities holds strictly. We denote the strict core by $SC(X,U)$. 
\subsection{Fixed Preallocations and the Strict Core}
Our method proceeds by identifying allocations $Y \in SC(X,U)$ with fixed points of an operator on preallocations, generalizing the construction in Echenique and Yenmez. 

For each agent $a \in A$, we define 

\[
V(\pre, a) = \{Z \in 2^{X_a}: \exists Y \in 2^X \, s.t. \,  Y_a = Z, \, Y_b \geq_b \pre(b) \: \forall b \in d(Y)\setminus\{a\} \}
\]

Intuitively, $V(\pre, a)$ is the \emph{possibility set} for an agent $a$ at a preallocation $\pre$. 
It contains all sets of contracts naming $a$ that are part of a larger economy $Y$ where every other agent $b \in d(Y)$ weakly prefers their contracts under $Y$ to their contracts under the pre-allocation $\pre$.  

Next, we define an operator $T: \prealloc \to \prealloc$ by setting $T \pre(a) = \max V(\pre, a)$, where the maximum is taken under the preference relation $\geq_a$ for each $a \in A$.
Note that $\emptyset_a \in V(\pre,a)$ for any $\pre \in \prealloc$, so $T$ is well-defined. Let $\fix(T)$ denote the fixed points of $T$. Define $\fecon(T) = \{Y \in 2^X: \, \pre_Y \in \fix(T)\}$, the collection of allocations $Y$ whose corresponding preallocation $\pre_Y$ is fixed by $T$.

Before our first result, we note a simple fact: if $\pre \in \coordpre$, then $\pre(a) \in V(\pre, a)$.
To see this, note that $\pre \in \coordpre$ means that there exists $Y \sub X$ with $\pre = \pre_Y$. 
Then $Y$ is an allocation satisfying the conditions in $V(\pre_Y,a)$, so that $Y_a  = \pre(a) \in V(\pre,a)$. 
With the definitions above, we have a simple result
\begin{lemma} $\fecon(T) = \strcore$
\end{lemma}
\prf
First, suppose that $Y \not \in \strcore$. Then by definition, there exists some blocking allocation $\emptyset \not = Z \sub X$ such $Z_b \geq_b Y_b$ for all $b \in d(Z)$.
Let $a \in d(Z)$ be such that the inequality above is strict, and consider $\pre = \pre_Y$. 
In particular, $Z_b \geq_b \pre_Y(b)$ for all $b \in d(Z) \setminus \{a\}$, so that $Z_a \in V(\pre_Y, a)$.
Then $T \pre_Y(a) = \max V(\pre_Y, a) \geq_a Z_a >_a Y_a = \pre_Y(a)$, so $T\pre_Y(a) \not = \pre_Y(a)$, and $Y \not \in \fecon(T)$.  

Suppose, conversely, that $Y \not \in \fecon(T)$ so that $\pre_Y \not \in \fix(T)$.  
Then there exists an agent $a \in A$ such that $T\pre_Y(a) = Z_a \not = \pre_Y(a)$ for some allocation $Z$.  
By the definition of $V(\pre_Y,a)$, we have $Z_b \geq_b \pre_Y(b) = Y_b$ for $b \in d(Z) \setminus \{a\}$.
We know $\pre_Y$ is coordinated, so by the simple fact above $\pre_Y(a) \in V(\pre_Y,a)$.
Then $Z_a = T \pre_Y(a) >_a \pre_Y(a) =  Y_a$. Then $Z$ is a blocking coalition for $Y$, so $Y \not \in \strcore$.  
\eprf
Thus, we have identified $\strcore$ with the set of \emph{coordinated} preallocations $\pre \in \coordpre$ such that $T \pre = \pre$. 
This result shows that an algorithm that finds all $\pre \in \fix(T)$ will also find all strict core matchings. 
However, if there are \emph{uncoordinated} preallocations that are also fixed by $T$, such an algorithm may return extraneous solutions not associated with any strict core matching.
Fortunately, there are no such prealloacations, as our next result shows
\begin{prop} $\fix(T) \sub \coordpre$
\end{prop}
\prf
\eprf
Note that this result strictly generalizes the analogous results in (Echenique, Kojima). 
\subsection{The Lattice of Fixed Preallocations}
In this section, we generalize constructions from Echenqiue and Yenmez showing the the fixed points of the squared operator $T^2$ form a lattice. 
First, we define a natural partial order on the set of preallocations $\prealloc$.

Say that $\pre > \pre'$ if and only if $\pre(a) \geq_a \pre'(a)$ for all $a \in A$, where at least one of these inequalities \emph{holds strictly}. 
Thus, we write $\pre \geq \pre'$ if and only if $\pre > \pre'$ or $\pre = \pre'$. 
This is a product order on a product space, which makes $\prealloc$ into a complete lattice (cite)(footnote about joins and meets). 
Next, we give a sequence of results concerning the operator $T$ and its fixed points. 
These results are an almost direct extension Lemmas 4.x through 4.y of Echenique and Yenmez.
For convenience, we reproduce the first result in our framework - the rest follow from work in Echenique in and Yenmez. 
\begin{lemma} $T$ is an antitone operator
\end{lemma}
\prf
Let $\pre \leq \pre'$ be preallocations.
Fix $a \in A$, and let $Z \in V(\pre',a)$.
Then there is an allocation $Y \sub X$ with $Y_a = Z$ such that $Y_b \geq_b \pre'(b)$ for $b \in d(Y) \setminus \{a\}$.
Then $Y_b \geq_b \pre'(b) \geq_b \pre_b$ also for all such agents, so we also have $Z \in V(\pre,a)$.
Then $V(\pre,a) \supseteq V(\pre',a)$, so that $T\pre(a) \geq T\pre'(a)$. $a$ was arbitrary, so $T\pre \geq T\pre'$ under our partial order. 
\eprf
The following lemmas follow exactly as in Echenique and Yenmez, using the antitonicity of $T$. 
\begin{cor} $T^2$ is isotone, and $\fix(T^2)$ is a non-empty complete lattice. 
\end{cor}
\begin{lemma} No two preallocations $\pre$ and $\pre'$ can be compared under the partial order on $\prealloc$.
\end{lemma}
\begin{lemma} There exist preallocations $\ol{\pre}$ and $\ul{\pre}$ such that for all $\pre \in \fix(T)$, we have $\ol{\pre} \geq \pre \geq \ul{\pre}$. Moreover, if $\pre = \ol{\pre}$ or $\pre = \ul{\pre}$, then $\fix(T) = \{\pre\}$. 
\end{lemma}
\subsection{Exploiting Lattice Geometry to Find All Strict Core Allocations}
In this section, we improve upon the algorithm Echenique (2003) and Echenique and Yenmez (2013), showing how to fully exploit the geometry of the lattice $\fix(T)$ to more quickly find all strict core matchings. 










\subsubsection{Discussion}
As can be seen, the construction above is very general, requiring only that the fixed points we want to find are bounded by a lattice, and that we have a way to initialize new problems on sublattices whose solution sets bound the solution set of our original problem. 

\section{Finding All Nash Equilibria in Games with Strategic Complements}
\subsection{Model and Notation}









\end{document}

